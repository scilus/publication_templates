\documentclass[8pt]{extarticle}

\usepackage{array}
\usepackage{titling}  % needs to be loaded before authblk
\usepackage{authblk}
\usepackage{booktabs}       % professional-quality tables
\usepackage[utf8]{inputenc} % allow utf-8 input
\usepackage[T1]{fontenc}    % use 8-bit T1 fonts
\usepackage[english]{babel}
\usepackage[natbib,style=numeric,natbib=true,maxbibnames=1,maxcitenames=1,giveninits=true,sorting=none]{biblatex}
\usepackage{csquotes}
\usepackage[inline]{enumitem}
\usepackage{float}
\usepackage{geometry}
\usepackage{graphicx}
\usepackage{hyperref}       % hyperlinks
\usepackage{lipsum}
\usepackage{multirow}
% Set paragraph skip and indent
\usepackage[skip=0pt,indent=4pt]{parskip}
\usepackage{url}            % simple URL typesetting
\usepackage{siunitx}
\usepackage{textcomp}       % for \textdegree macro
\usepackage{tikz}
\usepackage{wrapfig}
\usepackage{xcolor}
\usepackage{soul}
\usepackage{xspace, setspace} % defines a command \xspace that guesses whether there should have been a space after it, and if so introduces that space; useful to replace any space "eaten" by the TeX command decoder.

% Allow line breaks in column names
\newcommand{\head}[2]{\multicolumn{1}{>{\centering\arraybackslash}p{#1}}{\textbf{#2}}}
% e.g. \textbf{col_name} & \head{0.1\linewidth}{\textbf{Mean curvature}} & \textbf{col_name} 

% Modify the paragraph command to reduce the spacing
\makeatletter
\renewcommand\paragraph{
  \@startsection{paragraph}{4}{\z@}%
  %{3.25ex \@plus1ex \@minus.2ex}%
  {3pt plus 1pt minus 1pt}%
  {-1em}%
  {\normalfont\normalsize\bfseries}}
\makeatother

% Change the author and affiliation font size and the baselineskip
\renewcommand{\Authfont}{\bfseries\fontsize{10pt}{10pt}\selectfont}
\renewcommand{\Affilfont}{\normalfont\fontsize{8pt}{10pt}\selectfont}
% Change the affiliation separation (the separation between the author and affiliation lines)
\setlength{\affilsep}{0pt}

% Customize the title command: remove all vertical skip spaces
\makeatletter
\def\@maketitle{%
  \newpage
%  \null% DELETED
%  \vskip 2em% DELETED
  \begin{center}%
  \let \footnote \thanks
    {\LARGE \@title \par}%
%    \vskip 1.5em%  DELETED
    {\large
%      \lineskip .5em%  DELETED
      \begin{tabular}[t]{c}%
        \@author
      \end{tabular}\par}%
%    \vskip 1em%  DELETED
    {\large \@date}%
  \end{center}%
  \par
%  \vskip 1.5em}
  \vskip 0.005em}
\makeatother

% Remove top/bottom space around figures
\setlength\intextsep{0pt}
\setlength\floatsep{-20pt}

% Modify the space above caption
\setlength{\abovecaptionskip}{4pt plus 3pt minus 2pt}

% Rename the reference paragraph name
\addbibresource{./bibliography/bibliography.bib}
% Remove the name initials in bibliography
\DeclareNameAlias{default}{labelname}\DeclareNameAlias{sortname}{default}
% Remove indents and line breaks in bibliogrpahy items
\defbibenvironment{bibliography}
  {\noindent}
  {\unspace}
  {\printtext[labelnumberwidth]{%
     \printfield{labelprefix}%
     \printfield{labelnumber}}%
   \addspace}
\renewbibmacro*{finentry}{\finentry\addspace}

% Abstract preparation guidelines
% https://www.ismrm.org/workshops/2022/Diffusion/call.htm
%
% Abstracts are limited to one page in length (21.59 cm x 27.94 cm / 8.5"x11"), including all images, tables, graphs and references. Font size should be no smaller than 8 pt.
% 
% SUGGESTED FORMAT: The abstract should be formatted in one column, containing all of the following sections that answer the questions below:
% 
% Abstract title
% Author list
% Target audience: "Who will benefit from this information?"
% Purpose: "Why was this study/research performed?"
% Methods: "How has this problem been studied?"
% Results: "Principal data and statistical analysis"
% Discussion: "What is the interpretation of the data?"
% Conclusion: "What is the relevance to clinical practice or future research?"
% References: (optional)

\title{\textbf{\huge One page abstract model title}}

\newcommand\repeatthanks[1][\value{footnote}]{\footnotemark[#1]}
\author[1,2]{Jon Haitz Legarreta}
\author[3]{Laurent Petit}
\author[2]{Pierre-Marc Jodoin\thanks{Co-senior author. These authors contributed equally.}}
\author[1]{Maxime Descoteaux\repeatthanks}

\affil[1]{Sherbrooke Connectivity Imaging Laboratory (SCIL), Universit\'{e} de Sherbrooke, Sherbrooke, Canada}
\affil[2]{Videos \& Images Theory and Analytics Laboratory (VITAL), Universit\'{e} de Sherbrooke, Sherbrooke, Canada}
\affil[3]{Universit\'{e} Bordeaux, CNRS, CEA, IMN, GIN, UMR 5293, F-33000 Bordeaux, France}

% Do not print the date
\date{\vspace{-0.25in}}
% Remove spacing around date
\predate{}
\postdate{}

\setlength{\droptitle}{-0.5in}

\geometry{letterpaper, margin=0.25in}

% Turn off page numbering
\pagenumbering{gobble}

\begin{document}

\maketitle

%% Purpose
\paragraph{PURPOSE}
\label{par:purpose}

\begin{wrapfigure}{r}{0.15\linewidth}
\centering
\includegraphics[scale=0.95, width=0.95\linewidth]{example-image}
\caption{\label{fig:fig1label}Caption1.}
\end{wrapfigure}

\lipsum[1]

This is motivated by the work in \citep{Maier-Hein:NatureComm:2017}.

Figure \ref{fig:fig1label} shows the principle.

%% Methods
\paragraph{METHODS}
\label{par:methods}

\begin{wrapfigure}{r}{0.35\linewidth}
\centering
\includegraphics[scale=0.95, width=0.95\linewidth]{example-image-a}
\caption{\label{fig:fig2label}Caption2.}
\end{wrapfigure}

\lipsum[2-3]

Some early works in local modelling by \citet{Descoteaux:MRM:2007} gave rise many years later to other tractography works, such as \citep{Legarreta:MIA:2021}.

Figure \ref{fig:fig2label} is yet another insightful figure.

%% Results
\paragraph{RESULTS}
\label{par:results}

\lipsum[4]

Figure \ref{fig:fig3label} shows the local orientation field and the resulting tractogram. Table \ref{tab:table1label} shows quantitative results.

\begin{wrapfigure}{l}{0.45\linewidth}
\centering
\begin{tabular}{cc}
\includegraphics[scale=0.95, width=0.45\linewidth]{example-image-b} &
\includegraphics[scale=0.95, width=0.45\linewidth]{example-image-c} \\
\end{tabular}
\caption{\label{fig:fig3label}Caption3.}
\end{wrapfigure}

\lipsum[5-6]

\begin{wraptable}{r}{0.355\linewidth}
\centering
\caption{\label{tab:table1label}Caption4.}
\centering
\begin{tabular}{c|c|c|c}
\toprule
\textbf{Column1} & \textbf{Column2} & \textbf{Column3} & \textbf{Column4} \\
\midrule
Method1 & ParamVal1 & \num{94.3} (\num{4.5}) & \num{14.7} (\num{6.9}) \\
Method2 & ParamVal2 & \num{96.5} (\num{2.1}) & \num{11.5} (\num{4.2}) \\
\midrule
Method3 & ParamVal1 & \num{98.7} (\num{1.9}) & \num{8.7} (\num{1.4}) \\
Method4 & ParamVal2 & \num{99.8} (\num{1.1}) & \num{1.4} (\num{0.3}) \\
\bottomrule
\end{tabular}
\end{wraptable}

\lipsum[7]

%% Discussion
\paragraph{DISCUSSION}
\label{par:discussion}

\lipsum[8-9]

%% Conclusion
\paragraph{CONCLUSION}
\label{par:conclusion}

\lipsum[10]

%% Bibliography
\paragraph{REFERENCES} \printbibliography[heading=none]

\end{document}
